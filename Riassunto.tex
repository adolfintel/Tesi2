% !TeX spellcheck = it_IT

\documentclass[a4paper]{article}

\usepackage[english,italian]{babel}
\usepackage[a4paper,top=1cm,bottom=2cm,outer=1cm,inner=1cm,verbose,headheight=1cm,heightrounded]{geometry}
\usepackage[pdftitle={OpenLDAT - Un sistema di misurazione di metriche di latenza dei display (Riassunto della tesi)},pdfauthor={Federico Dossena},hidelinks=true]{hyperref}

\title{OpenLDAT\\Un sistema di misurazione di metriche di latenza dei display}
\author{Federico Dossena\\Matr. 909390}
\date{Anno Accademico 2020/2021}

\begin{document}
\maketitle

\sloppy
\setlength{\parskip}{8pt}
\setlength{\parindent}{0pt}
\section*{Riassunto}
Il progetto OpenLDAT (abbreviazione di \textit{Open Latency Display and Analysis Tool}) si pone l'obiettivo di realizzare un dispositivo e un software da utilizzare per misurare alcune metriche di latenza dei moderni display LCD. Tra le varie metriche misurabili, che saranno elencate in seguito, di particolare interesse è la latenza totale del sistema, ovvero il tempo che intercorre tra un'azione nel mondo fisico, come la pressione di un tasto del mouse, e la visualizzazione del risultato sullo schermo; questo valore dipende da molti fattori che vengono elencati e analizzati nella tesi, ma i principali contributori sono il display utilizzato, l'applicazione e la velocità dell'hardware. Altri nomi utilizzati per questa metrica sono: \textit{click-to-photon response time}, \textit{end-to-end latency} e \textit{input lag} (più diffuso ma impreciso).

Attualmente non esistono sul mercato dispositivi immediatamente acquistabili per questo scopo; la misura viene tradizionalmente eseguita utilizzando un mouse modificato per far accendere un LED quando viene premuto il pulsante sinistro e una telecamera ad alta velocità per registrare il LED e il monitor che si sta testando mentre si esegue un videogioco che risponde alla pressione del tasto con un impulso immediatamente visibile (ad esempio, il \textit{muzzle flash} di un'arma). Questo metodo è molto inefficiente, poiché richiede un'analisi manuale della cattura della telecamera, ed è anche relativamente impreciso poiché si è limitati dalla risoluzione temporale della telecamera ad alta velocità (tipicamente tra 480 e 1000Hz); l'implementazione e il framerate del videogioco utilizzato influiscono anch'essi sulla misura e la rendono difficile da replicare. OpenLDAT è in grado di eseguire questa misurazione in modo del tutto automatizzato con un test integrato, o in modo manuale utilizzando un mouse modificato e un videogioco, e lasciando al software l'analisi e la generazione del risultato.

Il dispositivo OpenLDAT è composto da un microcontroller ATmega 32U4, un fototransistor ALS-PT19, delle resistenze per variarne il livello di gain, un LED di stato, un connettore per un mouse o pulsante esterno opzionale, e un PCB personalizzato che connette tutti i componenti. Il microcontroller ha il compito di campionare il sensore, di comunicare con il PC, e di catturare i click dall'esterno o generarli automaticamente se richiesto. Il dispositivo è montato all'interno di un piccolo case stampato in 3D per proteggerlo. Il firmware sul dispositivo è ottimizzato per un campionamento rapido e regolare a diverse velocità (fino a 30kHz a 10 bit), con 4 livelli di gain per il sensore controllabili dal software, ed è facilmente estensibile. La comunicazione con il PC avviene tramite USB, nello specifico il dispositivo si mostra al PC come un mouse (USB HID) e una porta seriale (USB CDC Serial): in questo modo, i click che il dispositivo invia al PC vengono trattati come se fosse un vero mouse, e non è necessario installare alcun driver per permettere all'applicazione di funzionare.

L'applicazione OpenLDAT è un software multipiattaforma sviluppato utilizzando Java SE e OpenGL che implementa tutti i test che possono essere eseguiti con il dispositivo, oltre ad un'interfaccia grafica che permette di avviare e configurare i test, e mostrarne i risultati in modo semplice e intuitivo. I test implementati sono i seguenti:\begin{itemize}
	\item \textbf{Latenza totale del sistema (test automatico)}: utilizza il dispositivo per generare automaticamente click a intervalli regolari, che vengono ricevuti dall'applicazione stessa, che genera dei flash sullo schermo in risposta; il software analizza la cattura del sensore per misurare il tempo che intercorre tra l'invio dei click da parte del dispositivo e l'inizio del flash sullo schermo. Il test è in grado di simulare diversi scenari riscontrabili in un videogioco. Durante il test, il LED di stato lampeggia con i click per consentire la validazione dei risultati usando una telecamera ad alta velocità
	\item \textbf{Latenza totale del sistema (test manuale)}: permette di misurare la latenza utilizzando virtualmente qualsiasi applicazione (tipicamente un videogioco), potenzialmente anche in esecuzione su un'altra macchina, utilizzando un mouse opportunamente modificato per essere connesso al dispositivo, oppure utilizzando il dispositivo stesso per generare click periodici come nel test automatico. Il test prosegue fino a quando non viene interrotto, e i risultati sono immediatamente visibili all'utente
	\item \textbf{Rilevamento di PWM e noise}: rileva la presenza di retroilluminazione PWM (\textit{Pulse-Width Modulation}) e altri tipi di rumore, e ne mostrarne la frequenza (se presente). Tutti i test nell'applicazione devono gestire la presenza di questi disturbi che causano ``buchi'' nel segnale che viene catturato, e lo fanno utilizzando dei filtri software e degli algoritmi realizzati appositamente
	\item \textbf{Tempo di risposta dei pixel}: misura il tempo che i pixel impiegano per eseguire la transizione tra diverse sfumature di grigio. Esistono vari modi per misurarlo, e ogni produttore lo misura diversamente; il metodo proposto in questa tesi è una variazione del metodo proposto da VESA che consiste nel misurare i tempi di risposta tra coppie di sfumature di grigio, considerando come tempo di risposta il tempo necessario per eseguire l'intervallo 10-90\% della transizione
	\item \textbf{Overdrive dei pixel}: misura l'errore commesso dai pixel durante le transizioni tra sfumature di grigio. Non esiste uno standard per misurarlo, per cui in questa tesi ne vengono proposti due: errore percentuale assoluto ed errore percentuale relativo
	\item \textbf{Rilevamento del microstuttering}: rileva la perdita o duplicazione di fotogrammi, e lo fa alternando fotogrammi bianchi e neri e misurando se ci sono variazioni nei tempi di ogni colore. Questo fenomeno si verifica su alcuni display quando il segnale in input non corrisponde con il refresh rate nativo del display (\textit{overclock})
	\item \textbf{Light to Sound}: permette di utilizzare il sensore del dispositivo per ascoltare fonti luminose, vederne la forma d'onda come su un oscilloscopio, e rilevarne la frequenza (se presente). Questa funzione consente di trovare fonti di disturbi luminosi, come lampade LED con un filtraggio inadeguato dell'alimentazione
\end{itemize}

Infine, il dispositivo e l'applicazione sono stati utilizzati per testare display di tipi e periodi diversi, oltre 20 in totale. A causa delle condizioni causate dalla pandemia nei primi mesi del 2021, alcuni test sono stati eseguiti dall'autore della tesi, mentre altri sono stati eseguiti da terzi che hanno ricevuto un prototipo del dispositivo; le condizioni di test sono state controllate il più possibile al fine di garantire la qualità dei risultati ottenuti e assicurarsi che fossero confrontabili, ma sarebbe comunque ideale ripeterli in condizioni completamente controllate. Sono stati eseguiti i seguenti test:\begin{itemize}
	\item \textbf{Latenza totale del sistema}: confronto tra display, confronto tra diverse combinazioni hardware e software (sistemi operativi, driver, GPU), confronto tra applicazioni, validazione dei risultati con telecamera ad alta velocità (1000Hz)
	\item \textbf{Rilevamento di PWM e noise}: confronto tra display e analisi dei tipi di rumore rilevati
	\item \textbf{Tempo di risposta e overdrive dei pixel}: confronto tra display con e senza overdrive
	\item \textbf{Rilevamento del microstuttering}: confronto tra display a refresh rate nativo e in overclock
\end{itemize}

Evoluzioni future del progetto includono l'aggiunta di un colorimetro per poter eseguire anche test sull'accuratezza dei colori (anche se esistono già dispositivi per questo scopo), l'utilizzo di un ADC a risoluzione maggiore, migliorie all'interfaccia dell'applicazione, il porting dell'applicazione su altre piattaforme (attualmente supporta Windows, GNU/Linux e MacOS), la creazione di un sito dove gli utenti possono caricare i risultati dei propri display, e dato che nessuno l'ha fatto finora, la commercializzazione del dispositivo.

Il progetto OpenLDAT è completamente libero e distribuito su licenza GNU GPL v3.

\end{document}
