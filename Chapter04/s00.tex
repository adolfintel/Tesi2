Questo capitolo si concentra sull'applicazione OpenLDAT per PC, illustrandone i requisiti, le tecnologie utilizzate, e i dettagli del funzionamento, con un focus particolare agli algoritmi che analizzano i dati raccolti dal dispositivo.

I requisiti principali per l'applicazione OpenLDAT sono i seguenti:
\begin{itemize}
	\item Implementare dei test che utilizzano il dispositivo OpenLDAT per raccogliere dati e li analizzano per estrarre informazioni
	\item I test implementati devono essere il più possibile rappresentativi dell'utilizzo reale del sistema
	\item Funzionare su più sistemi operativi, almeno GNU/Linux, Windows e MacOS
	\item Essere facile da installare e utilizzare anche per un utente poco esperto
\end{itemize}

Prima di iniziare lo sviluppo sono state prese in considerazione più tecnologie per implementare l'applicazione: inizialmente sono stati fatti degli esperimenti con Electron, che oggi è molto popolare per sviluppare applicazioni desktop multipiattaforma, ma si sono presto presentate difficoltà dovute al fatto che l'engine Chromium al suo interno introduce esso stesso un notevole ritardo di input, per cui i test non sarebbero stati rappresentativi; la scelta finale è quindi ricaduta su Java SE con OpenGL, scelta che consente molta più libertà di implementazione, e di realizzare quindi test più rappresentativi di un utilizzo reale.

L'applicazione è strutturata in quattro parti principali più alcune classi di utilità:
\begin{itemize}
	\item \textbf{Comunicazione con il dispositivo} (package \texttt{com.dosse.openldat.device}): è sostanzialmente un driver per il dispositivo OpenLDAT che ha il compito di rilevare il dispositivo e consentire al resto dell'applicazione di utilizzarlo in modo semplice, senza preoccuparsi dei dettagli di comunicazione
	\item \textbf{Processamento di base} (package \texttt{com.dosse.openldat.processing}): fornisce un insieme di strumenti per memorizzare i dati in arrivo dal dispositivo e dei filtri che possono essere applicati su di essi (ad esempio, FFT)
	\item \textbf{I test} (package \texttt{com.dosse.openldat.tests}): questa è la parte principale dell'applicazione e implementa tutti i test che sono stati sviluppati per OpenLDAT, oltre a due backend grafiche per disegnare sullo schermo con e senza OpenGL
	\item \textbf{La GUI} (package \texttt{com.dosse.openldat.ui}): implementa l'interfaccia grafica per l'applicazione che permette di configurare ed eseguire i test e visualizzarne in modo grafico i risultati
\end{itemize}

Oltre a Java SE, per lo sviluppo dell'applicazione sono state utilizzate anche le seguenti librerie:
\begin{itemize}
	\item \textbf{jSerialComm} \footnote{\href{https://github.com/Fazecast/jSerialComm}{https://github.com/Fazecast/jSerialComm}}: una libreria che consente ad applicazioni Java di interfacciarsi con dispositivi seriali
	\item \textbf{LWJGL} \footnote{\href{https://www.lwjgl.org/}{https://www.lwjgl.org/}}: una libreria per lo sviluppo di applicazioni OpenGL in Java, utilizzata anche da giochi commerciali come Minecraft
	\item \textbf{JTransforms} \footnote{\href{https://github.com/wendykierp/JTransforms}{https://github.com/wendykierp/JTransforms}}: utilizzata per realizzare il filtro FFT
\end{itemize}

Tutti i file relativi all'applicazione e al packaging per le varie piattaforme sono presenti nella cartella \texttt{App} all'interno del repository. Il progetto \texttt{OpenLDAT} al suo interno può essere caricato in NetBeans IDE.

Nelle sezioni successive verranno discussi approfonditamente tutti gli aspetti dell'applicazione.
