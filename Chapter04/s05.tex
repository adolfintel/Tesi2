\section{Packaging dell'applicazione}
Per consentire un facile utilizo dell'applicazione, soprattutto da utenti inesperti, è necessario creare dei pacchetti binari per le varie piattaforme. Sono stati scelti i seguenti target:\begin{itemize}
	\item Windows 10 x64
	\item GNU/Linux amd64 (diverse distribuzioni popolari)
	\item MacOS Intel 64 bit
\end{itemize}
Essendo il codice multipiattaforma, è possibile utilizzarlo anche su altre piattaforme come BSD o Windows a 32 bit, tuttavia queste non sono state testate e non si garantisce il corretto funzionamento.

Tutti i file e le istruzioni necessarie per il packaging sulle piattaforme target sono presenti nella cartella \texttt{Packaging stuff} all'interno della cartella dell'applicazione.

\subsection{Windows}
Il pacchetto di installazione per Windows è stato realizzato utilizzando Inno Setup\footnote{\href{https://jrsoftware.org/isinfo.php}{https://jrsoftware.org/isinfo.php}}. All'interno del package sono presenti l'applicazione OpenLDAT con le relative librerie, il runtime Java (OpenJDK 11 JRE\footnote{\href{https://adoptopenjdk.net/releases.html}{https://adoptopenjdk.net/releases.html}}), e un launcher eseguibile per Windows realizzato con Launch4J\footnote{\href{https://launch4j.sourceforge.net/}{https://launch4j.sourceforge.net/}}.

La procedura per realizzare il package è la seguente:\begin{itemize}
	\item Fare una copia della cartella \texttt{Windows-InnoSetup}
	\item Eseguire la build del progetto OpenLDAT da NetBeans IDE
	\item Dalla cartella \texttt{dist} nel progetto appena compilato, copiare \texttt{OpenLDAT.jar} e la cartella \texttt{lib} nella cartella \texttt{openldat}
	\item Scaricare il file zip di OpenJDK JRE x64 versione 11 o superiore dal relativo sito
	\item All'interno dello zip scaricato estrarre i file del runtime (cartelle \texttt{bin}, \texttt{lib}, eccetera) nella cartella \texttt{jre}
	\item Utilizzando Launch4J, eseguire la build del file \texttt{launcher.xml}
	\item Utilizzando Inno Setup 6, eseguire la build del file \texttt{setup.iss}. Al termine verrà generato un file chiamato \texttt{OpenLDAT\_Setup.exe} che può essere distribuito. Si consiglia di firmarlo digitalmente
\end{itemize}

\subsection{GNU/Linux}
Il pacchetto per GNU/Linux è distribuito nel formato AppImage\footnote{\href{https://appimage.org/}{https://appimage.org/}}, che non richiede installazione. All'interno del package sono presenti l'applicazione OpenLDAT con le relative librerie e il runtime Java (OpenJDK 11 JRE).

La procedura per realizzare il package è la seguente:\begin{itemize}
	\item Fare una copia della cartella \texttt{Linux-AppImage} ed entrare nella cartella \texttt{OpenLDAT.AppDir}
	\item Eseguire la build del progetto OpenLDAT da NetBeans IDE
	\item Dalla cartella \texttt{dist} nel progetto appena compilato, copiare \texttt{OpenLDAT.jar} e la cartella \texttt{lib} nella cartella \texttt{openldat}
	\item Scaricare il file zip di OpenJDK JRE x64 versione 11 o superiore dal relativo sito
	\item All'interno dello zip scaricato estrarre i file del runtime (cartelle \texttt{bin}, \texttt{lib}, eccetera) nella cartella \texttt{jre}
	\item Tornare al livello superiore ed eseguire il seguente comando:\begin{verbatim}
		appimagetool OpenLDAT.AppDir
	\end{verbatim}
	Al termine verrà generato un file chiamato \texttt{OpenLDAT-x86\_64.AppImage} che può essere distribuito
\end{itemize}

\subsection{MacOS}
Il pacchetto per MacOS è distribuito sotto forma di immagine DMG contenente un'applicazione nel formato standard .app, la quale contiene al suo interno l'applicazione OpenLDAT con le relative librerie, il runtime Java (OpenJDK 11 JRE), e un wrapper per consentire l'avvio di applicazioni Java su MacOS\footnote{\href{https://github.com/tofi86/universalJavaApplicationStub}{https://github.com/tofi86/universalJavaApplicationStub}}.

Nota: il testing su questa piattaforma è stato estremamente limitato e non è stato eseguito dall'autore di questa tesi.

La procedura per realizzare il package è la seguente:\begin{itemize}
	\item Fare una copia della cartella \texttt{Mac}
	\item Eseguire la build del progetto OpenLDAT da NetBeans IDE
	\item Dalla cartella \texttt{dist} nel progetto appena compilato, copiare \texttt{OpenLDAT.jar} e la cartella \texttt{lib} nella cartella \texttt{openldat}
	\item Scaricare il file zip di OpenJDK JRE x64 versione 11 dal relativo sito\\
	Attenzione: La versione 11 è l'unica supportata da questo sistema di packaging al momento
	\item All'interno dello zip scaricato estrarre i file del runtime (cartelle \texttt{bin}, \texttt{lib}, eccetera) nella cartella \texttt{jre/openjdk-11-jre}
	\item Eseguire lo script \texttt{makeapp}. Al termine verrà generata una cartella \texttt{OpenLDAT.app}
	\item Utilizzare Disk Utility per creare un'immagine DMG compressa contenente \texttt{OpenLDAT.app} che può essere distribuita
\end{itemize}

Questo conclude il capitolo sull'applicazione OpenLDAT per PC. Nel capitolo successivo verranno mostrati e discussi alcuni risultati raccolti con l'applicazione e il dispositivo in diversi scenari.
