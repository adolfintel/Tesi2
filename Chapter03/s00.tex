In questo capitolo viene introdotto il dispositivo fisico OpenLDAT, descrivendone i requisiti, le caratteristiche, il funzionamento dell'hardware e del firmware, e i passi necessari per realizzarlo a partire da un elenco di componenti.

Il dispositivo OpenLDAT ha sostanzialmente quattro compiti:
\begin{itemize}
	\item Campionare nel modo più regolare e veloce possibile un sensore di luminosità
	\item Generare dei click (automaticamente o manualmente a seconda del test), facendo finta di essere un mouse
	\item Far lampeggiare un LED sul dispositivo quando vengono generati click, in modo che le misurazioni sulla latenza siano verificabili usando una telecamera ad alta velocità
	\item Gestire la comunicazione con il PC, ricevendo comandi e inviando i dati del sensore e del mouse virtuale
\end{itemize}

Un obiettivo principale del progetto è anche far si che il dispositivo sia costruito usando componenti off-the-shelf, con costi relativamente bassi (molto meno di un colorimetro), senza bisogno di apparecchiature speciali per la sua costruzione. In altre parole, deve essere costruibile nel tipico laboratorio di un maker.

