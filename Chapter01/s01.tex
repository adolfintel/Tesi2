\section{Struttura della tesi}
Questa tesi è strutturata in questo modo:
\begin{itemize}
	\item \textbf{\autoref{chap:intro}}: Introduce tutti i concetti basilari, la terminologia e gli obiettivi del progetto OpenLDAT
	\item \textbf{\autoref{chap:statoarte}}: Presenta una panoramica sullo stato dell'arte, introducendo progetti simili o correlati, e in che modo differiscono da OpenLDAT
	\item \textbf{\autoref{chap:device}}: Si concentra sul dispositivo OpenLDAT fisico, spiegando il funzionamento dell'hardware e del firmware, i componenti utilizzati, e come realizzarlo in autonomia
	\item \textbf{\autoref{chap:app}}: Si concentra sul software OpenLDAT per PC, illustrandone le caratteristiche, gli algoritmi realizzati, le tecnologie utilizzate, e i dettagli implementativi
	\item \textbf{\autoref{chap:expdata}}: Presenta e analizza i risultati sperimentali ottenuti in diverse condizioni con il dispositivo e il software OpenLDAT
	\item \textbf{\autoref{chap:outro}}: Riassume il lavoro svolto e propone possibili evoluzioni per il progetto
\end{itemize} 
