\newglossaryentry{refreshrate}{
	name={Refresh rate},
	description={la frequenza con cui vengono aggiornate le immagini sullo schermo. Tipicamente è 60 Hz.}
}
\newglossaryentry{vrr}{
	name={VRR},
	description={Variable Refresh Rate, consente al display di variare il refresh rate entro un certo intervallo, consentendo alla GPU di inviare i fotogrammi quando sono pronti, riducendo il ritardo ed eliminando il tearing senza bisogno di VSync. Lo standard si chiama VESA Adaptive Sync, i nomi commerciali sono Nvidia G-Sync e AMD Freesync.}
}
\newglossaryentry{responsetime}{
	name={Tempo di risposta dei pixel},
	description={il tempo in millisecondi che un pixel impiega per transizionare tra un colore e un altro. Esistono vari metodi per misurarlo.}
}
\newglossaryentry{overdrive}{
	name={Overdrive},
	description={una tecnica utilizzata per ridurre i tempi di risposta dei pixel ``esagerando'' le transizioni (per esempio, se si deve passare da 32 a 64, se per un breve istante si passa a 128 e poi si torna a 64 la transizione sarà più rapida). Overdrive eccessivo può causare artefatti visibili.}
}
\newglossaryentry{ghosting}{
	name={Motion blur / Ghosting},
	description={scia visibile dietro agli oggetti in movimento causata dai tempi di risposta dei pixel.}
}
\newglossaryentry{inverseghosting}{
	name={Inverse ghosting},
	description={artefatto creato dall'eccessivo overdrive, simile al ghosting, ma di colore opposto.}
}
\newglossaryentry{pwm}{
	name={PWM},
	description={Pulse Width Modulation, lampeggio periodico della retroilluminazione per variare la luminosità.}
}
\newglossaryentry{strobing}{
	name={Strobing},
	description={disattivazione intenzionale della retroilluminazione, tipicamente per nascondere le transizioni dei pixel.}
}
\newglossaryentry{bfi}{
	name={Black Frame Insertion},
	description={tecnica adottata dai display a refresh rate molto alto (240Hz e più) per ridurre il motion blur percepito inserendo dei fotogrammi neri tra i fotogrammi di un segnale a frequenza inferiore.}
}
\newglossaryentry{dynamiccontrast}{
	name={Contrasto dinamico},
	description={tecnica adottata da alcuni display per variare il livello della retroilluminazione in base alla luminosità dell'immagine attualmente visualizzata. Permette di falsare un contrasto più elevato nei test.}
}
\newglossaryentry{hdrwcg}{
	name={HDR / WGC},
	description={High Dynamic Range e Wide Color Gamut. Indica la possibilità del display di visualizzare un range di luminosità e di colori più ampio rispetto al normale. In termini di marketing, tipicamente questo significa che il display è in grado di accettare un segnale RGB a 10 bit per canale nello spazio colore BT.2020. Esistono diversi standard che lo definiscono nel dettaglio, come HDR10 e Dolby Vision, e certificazioni come VESA DisplayHDR che indicano i livelli minimi di luminosità che un display deve avere per essere definito HDR.}
}
\newglossaryentry{localdimming}{
	name={Local Dimming},
	description={tecnica simile al contrasto dinamico, ma anziché variare la luminosità dell'intero schermo, può variare solo determinate zone controllando individualmente i LED della retrolluminazione. Tende a creare aloni attorno agli oggetti luminosi.}
}

%altro da aggiungere?