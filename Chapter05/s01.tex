\section{Input lag}
In questa sezione vengono utilizzati i test dell'applicazione OpenLDAT per determinare l'input lag di diversi display, e vedere come questo è influenzato dalla configurazione hardware e software.

Sono stati eseguiti i seguenti test:\begin{itemize}
	\item Confronto tra diversi display
	\item Confronto tra sistemi operativi, GPU e driver
	\item Confronto tra modalità di VSync nel test automatico
	\item Confronto tra diverse applicazioni utilizzando il test manuale
	\item Validazione dei risultati con telecamera ad alta velocità
\end{itemize}

\subsection{Confronto tra display}
In questo test è stato misurato il ritardo dei display utilizzando il test automatico dell'applicazione OpenLDAT. Il test è stato eseguito con e senza VSync, utilizzando il backend grafico OpenGL a risoluzione nativa. Al fine di avere dei risultati il più confrontabili possibile, tutti i test sono stati eseguiti su Windows 10 con una GPU Nvidia, ad eccezione dei laptop per cui è stato possibile utilizzare solo la grafica integrata Intel. In tutti i casi, i display sono stati collegati tramite un'interfaccia digitale (HDMI o DP), ad eccezione del Philips 105MB e dell'Octigen M19W che hanno solo un'ingresso VGA.

\begin{figure}[h!]
	\centering
	\pgfplotstableread[col sep=comma]{
		model,novsync,vsync
		Acer Predator XB271HU (165Hz G-Sync),6.3,32.5
		LG 27GL850-B (144Hz Freesync),6.4,36.9
		BenQ XL2420T (120Hz),9.2,44.1
		%Huawei Matebook D 2019 (60Hz Laptop),10.1,86.3 %dubito dell'accuratezza di questo risultato, sembra troppo basso il vsync off, probabilmente il tizio si è mosso durante il test
		Samsung C34H890 (100Hz Freesync),11.8,52.8
		Philips 105MB (85Hz CRT),13.1,64.2
		ASUS VP228HE (60Hz),13.1,86.8
		LG E2360 (60Hz),14.1,89.6
		Sony VAIO SVF1532C5E (60Hz Laptop),15.5,62.4
		ASUS VW228 (60Hz),16.0,88.5
		Thinkpad T480 2018 (60Hz Laptop),18.2,88.8
		Octigen M19W (60Hz),19.6,88.2
		BenQ GL2706PQ (60Hz),27.7,93.7
		AOC Q2770P (60Hz),29.8,95.8
		MacBook Pro 13" 2017 (60Hz Laptop),30.2,96.7
		Acer Swift 3 (60Hz Laptop),31.8,111.2
		Philips 32PFS4132 (60Hz TV),34.9,106.5
		Sharp LC-40FG3242E (60Hz TV),37.1,108.8
		Samsung P2770HD (60Hz TV),41.1,111.9
	}\dataset
	\begin{tikzpicture}
		\begin{axis}[xbar, bar width=8pt, y dir=reverse, ytick=data, yticklabels from table={\dataset}{model}, yticklabel style={text width=4cm, align=right}, table/y expr = \coordindex, nodes near coords, reverse legend, legend style={at={(0.5,-1.1cm)},anchor=north}, xlabel=Ritardo (ms), width=\textwidth-2cm, height=18cm, xmin=0, ymin=-1, ymax=18] %ymax messo a mano con il numero di display per migliore formattazione
			\addplot table[x=vsync] {\dataset};
			\addplot table[x=novsync] {\dataset};
			\legend{VSync On, VSync Off}
		\end{axis}
	\end{tikzpicture}
	\caption{Input lag dei display testati}
	\label{fig:inputlag_displays}
\end{figure}

Il grafico in figura \ref{fig:inputlag_displays} mostra i risultati che sono stati ottenuti sui display testati. Da questi dati possiamo notare alcune cose interessanti:\begin{itemize}
	\item Come atteso, i display ad alto refresh rate sono in cima alla classifica. Questi display sono stati progettati appositamente per questo scopo, tipicamente sacrificando la qualità dell'immagine per avere più velocità
	\item Alcuni dei display testati introducono un ritardo inferiore al tempo di un fotogramma, il che indica che questi display iniziano a visualizzare l'immagine sul pannello prima di averla ricevuta interamente, riducendo notevolmente il ritardo. Il "gradino" presente in corrispondenza al BenQ GL2706PQ delinea chiaramente il passaggio tra display che fanno questa "ottimizzazione" e display che invece attendono di ricevere l'intero fotogramma. Durante i test è emerso che alcuni display permettono di attivare o disattivare questa funzione a seconda delle proprie necessità, e tipicamente la chiamano "Direct Mode"
	\item I TV sono in fondo alla classifica con un certo distacco. Questo indica, oltre al fatto che un TV non è ideale come monitor, che il processore all'interno sta eseguendo un qualche tipo di elaborazione dell'immagine che in tutti i casi testati aggiungeva ritardi e non era disattivabile
	\item Il Sony VAIO SVF1532C5E ha un ritardo con VSync attivo molto inferiore alle aspettative. Questo outlier è probabilmente causato da qualche ottimizzazione specifica del driver
\end{itemize}

\subsection{Confronto tra sistemi operativi, GPU e driver}
In questo test è stato testato un singolo display (AOC Q2770P) su sistemi operativi, GPU e driver diversi, con e senza VSync, per determinare se c'è una differenza apprezzabile. I test sono stati eseguiti su Windows 10 e su Manjaro Linux 21 (KDE, X.11), su GPU Nvidia GTX 1080, AMD Radeon RX550, Intel UHD Graphics 620. La differenza di prestazioni tra le GPU testate non è particolarmente rilevante per il test automatico, in quanto il rendering è estremamente semplice e qualsiasi GPU lo esegue a centinaia o addirittura migliaia di FPS.

\begin{figure}[h!]
	\centering
	\pgfplotstableread[col sep=comma]{
		config,novsync,vsync
		Linux + AMD,28.7,86.1
		Linux + Nvidia (Proprietary),28.0,69.1
		Linux + Nvidia (Nouveau),44.3,86.0
		Linux + Intel,34.1,86.6
		Windows + AMD,29.6,93.9
		Windows + Nvidia,28.4,95.3
		Windows + Intel,31.3,103.4
	}\dataset
	\begin{tikzpicture}
		\begin{axis}[xbar, bar width=8pt, y dir=reverse, ytick=data, yticklabels from table={\dataset}{config}, yticklabel style={text width=3cm, align=right}, table/y expr = \coordindex, nodes near coords, reverse legend, legend style={at={(0.5,-1.1cm)},anchor=north}, xlabel=Ritardo (ms), width=\textwidth-3cm, height=9cm]
			\addplot table[x=vsync] {\dataset};
			\addplot table[x=novsync] {\dataset};
			\legend{VSync On, VSync Off}
		\end{axis}
	\end{tikzpicture}
	\caption{Input lag con combinazioni hardware/software diverse}
	\label{fig:inputlag_os}
\end{figure}

Il grafico in figura \ref{fig:inputlag_os} mostra i risultati del test. In generale i risultati sono più o meno simili tra loro, ma si possono notare dei comportamenti interessanti:\begin{itemize}
	\item Il driver proprietario di Nvidia per GNU/Linux, che è generalmente considerato uno dei peggiori dalla community, ha in realtà mostrato il ritardo di input migliore tra tutte le configurazioni testate, in particolar modo con il VSync attivo
	\item In generale, la GPU Intel mostra un ritardo leggermente maggiore rispetto ad AMD e Nvidia
	\item Il driver Nouveau mostra il ritardo peggiore con il VSync disattivato perché non supporta il power management sulla GPU testata, e quindi la scheda era bloccata nella sua modalità a risparmio energetico, renderizzando a un framerate notevolmente più basso
	\item In generale, Windows sembra mostrare un ritardo leggermente più elevato
\end{itemize}

\begin{figure}[h!]
	\centering
	\pgfplotstableread[col sep=comma]{
		rowid,config,novsync,vsync,vsyncalt
		0,GNU/Linux,28.7,86.1,48.3
		1,Windows,29.6,93.9,50.6
	}\dataset
	\begin{tikzpicture}
		\begin{axis}[xbar, ymin=-0.5,ymax=1.5, bar width=8pt, y dir=reverse, ytick=data, yticklabels from table={\dataset}{config}, yticklabel style={text width=2cm, align=right}, nodes near coords, reverse legend, legend style={at={(0.5,-1.1cm)},anchor=north}, xlabel=Ritardo (ms), width=\textwidth-3cm, height=5cm]
			\addplot table[y=rowid,x=vsync] {\dataset};
			\addplot table[y=rowid,x=novsync] {\dataset};
			\addplot table[y=rowid,x=vsyncalt] {\dataset};
			\legend{VSync On, VSync Off, VSync Alt}
		\end{axis}
	\end{tikzpicture}
	\caption{Confronto tra modalità di VSync su diversi sistemi operativi}
	\label{fig:inputlag_vsyncmodes}
\end{figure}

Il grafico in figura \ref{fig:inputlag_vsyncmodes} mostra un confronto tra le tre modalità di VSync del test automatico su Windows e GNU/Linux, entrambi testati con una GPU AMD. Si può osservare che Windows mostra un ritardo leggermente maggiore in tutte le modalità, nonostante sia generalmente ritenuto migliore da questo punto di vista. È possibile che questa discrepanza sia semplicamente dovuta ad OpenGL, che su Windows potrebbe essere per qualche motivo più lento, ma per determinarlo servirebbe un backend DirectX, che al momento non è implementato, ma potrebbe essere aggiunto in futuro.

\subsection{Confronto tra applicazioni}
In questo test è stato utilizzato il test manuale dell'applicazione OpenLDAT per determinare il ritardo su alcune applicazioni, per lo più giochi. Il test è stato eseguito su Windows 10 con una GPU Nvidia GTX 1080 e un display AOC Q2770P. Tutte le applicazioni sono state testate con la massima qualità grafica (ove applicabile) e il VSync disattivato per consentire all'applicazione di funzionare alla velocità massima possibile.

\begin{figure}[h!]
	\centering
	\pgfplotstableread[col sep=comma]{
		app,lag,emin,emax
		Mass Effect Legendary Ed. (2021),45.7,8.4,7.7
		Crysis (2007),51.9,6.7,8.6
		Doom Eternal (2020),52.8,4.6,2.4
		Unreal Tournament 2004 (2003),81.3,11.8,18.7
		Google Stadia 1080p60 (Fibra),121.4,34.4,119.6
		YouTube (Chromium),147.2,13.0,17.2
		Doom (1993),158.9,16.9,14.2
		Crysis Remastered (2020),182.5,17.8,23.5
	}\dataset
	\begin{tikzpicture}
		\begin{axis}[xbar, bar width=10pt, y dir=reverse, ytick=data, yticklabels from table={\dataset}{app}, yticklabel style={text width=4cm, align=right}, table/y expr = \coordindex, nodes near coords, xlabel=Ritardo (ms), width=\textwidth-3cm, height=9cm, ymin=-1, ymax=8] %ymax messo a mano con il numero di test per migliore formattazione
			\addplot[fill=gray, error bars/.cd, x dir = both, x explicit] table[x=lag, x error plus=emax, x error minus=emin] {\dataset};
		\end{axis}
	\end{tikzpicture}
	\caption{Input lag di alcune applicazioni. (150ms è generalmente considerato il limite accettabile per un videogioco)}
	\label{fig:inputlag_games}
\end{figure}

Il grafico in figura \ref{fig:inputlag_games} mostra i risultati del test. Si possono notare alcuni comportamenti interessanti: \begin{itemize}
	\item L'applicazione che ha mostrato il ritardo minore è stata Mass Effect Legendary Edition, che è stato inaspettato considerato che non è un gioco in cui il ritardo di input è rilevante
	\item Il divario nel ritardo di input tra Crysis (2007) e il suo remaster del 2020 ha dell'incredibile. Nel 2007 il gioco era infatti stato elogiato, oltre che per la parte tecnica, per il ritardo di input basso che lo rendeva giocabile anche a framerate bassi, anche intorno ai 18-25 FPS, che è il modo in cui quasi tutti lo giocarono all'epoca. Il remaster sembra aver dato un taglio netto al passato, ed è sostanzialmente ingiocabile sotto i 45-50 FPS
	\item Unreal Tournament 2004 sembra essere vittima di Windows 10 che lo costringe a funzionare in finestra con VSync nonostante venga richiesto il fullscreen esclusivo. Questo aumenta notevolmente il suo ritardo, che probabilmente sarebbe stato il minore tra tutti i giochi testati dato che il framerate superava i 700 FPS durante il test
	\item Doom (1993) registra un ritardo più alto rispetto a quello reale. Una breve ricerca ha rivelato che la causa è il fatto che il codice del gioco internamente funziona a 35 FPS, anche se il movimento e il rendering possono andare più velocemente
	\item Google Stadia, quando connesso a una connessione molto veloce (è stata utilizzata una connessione FTTH e un PC connesso via Ethernet), riesce a dare latenze tutto sommato accettabili, tuttavia ha mostrato la maggior variabilità tra tutte le applicazioni testate. Poiché è stato testato da terzi, non sono state svolte ulteriori indagini per determinare la causa di queste variazioni
\end{itemize}

\subsection{Validazione con telecamera ad alta velocità}
Come ultimo test è stata utilizzata una telecamera ad alta velocità (480 Hz) puntata al dispositivo OpenLDAT e l'area di schermo nelle immediate vicinanze, ed è stato eseguito un test manuale con un videogioco. Il video è stato poi analizzato manualmente per misurare il tempo che trascorreva tra l'accendersi del LED sul dispostivo e l'arrivo del flash sullo schermo, e i risultati sono stati confrontati con quelli misurati dall'applicazione. Il grafico in figura \ref{fig:inputlag_validation} mostra che le due misure sono essenzialmente identiche, a meno di un piccolo errore dovuto alla risoluzione temporale inferiore della telecamera rispetto al dispositivo.

\begin{figure}[h!]
	\centering
	\pgfplotstableread[col sep=comma]{
		Run,Time,CameraTime
		0,53.43300110742,54.1658
		1,40.0401439645624,39.5827
		2,39.0711517165006,39.5827
		3,40.9053156146193,41.666
		4,42.0473421926904,43.7493
		5,46.1309523809526,45.8326
		6,47.9651162790695,47.9159
		7,49.9377076411953,49.9992
		8,42.531838316723,43.7493
		9,37.3062015503898,37.4994
	}\dataset
	\begin{tikzpicture}
		\begin{axis}[xmin=0,xmax=9,ymin=0,ymax=70,width=.7\textwidth,ylabel=Ritardo (ms),xlabel=Run]
			\addplot[black] table[x=Run,y=Time] {\dataset};
			\addplot[red] table[x=Run,y=CameraTime] {\dataset};
		\end{axis}
	\end{tikzpicture}
	\caption{Confronto dei tempi misurati da OpenLDAT (nero) rispetto a una fotocamera ad alta velocità (rosso)}
	\label{fig:inputlag_validation}
\end{figure}
