\section{PWM e noise}
L'obiettivo di questo test è misurare la stabilità della retroilluminazione del display, in particolar modo per determinare la presenza e la frequenza di retroilluminazione PWM.

\begin{table}[h!]
	\centering
	\begin{tabular}{|c|c|c|} 
		\hline
		\textbf{Dispositivo} & \textbf{Frequenza PWM} & \textbf{Refresh rilevabile}  \\ 
		\hline
		Acer Predator XB271HU & No & No \\ \hline
		Acer Swift 3 & No & No \\ \hline
		AOC Q2770P & No & No \\ \hline
		ASUS VP228HE & No & Si \\ \hline
		ASUS VW228 & 240Hz & No \\ \hline
		BenQ GL2706PQ & No & Si \\ \hline
		BenQ XL2420T & No & No \\ \hline
		Huawei MateBook D 2019 & No & No \\ \hline
		iPhone 6S & No & No \\ \hline
		LG 27GL850-B & No & No \\ \hline
		LG E2360 & 240Hz & No \\ \hline
		MacBook Pro 13" 2017 & No & No \\ \hline
		Octigen M19W & N/A (CCFL) & No \\ \hline
		OnePlus 3T & 240Hz & Si \\ \hline
		OnePlus 7 Pro & No & Si \\ \hline
		Philips 32PFS4132 & 150Hz & No \\ \hline
		Philips 105MB & N/A (CRT) & Si \\ \hline
		Samsung C34H890 & No & Si \\ \hline
		Samsung P2770HD & 180Hz & No \\ \hline
		Sony VAIO SVF1532C5E & No & No \\ \hline
		Sharp Aquos LC-40FG3242E & 180Hz & No \\ \hline
		Thinkpad T480 2018 & No & No \\ \hline
	\end{tabular}
	\caption{\label{tab:pwm_list}PWM e noise dei display testati}
\end{table}

La tabella \ref{tab:pwm_list} mostra l'elenco dei display testati. Durante il test sono stati riscontrati alcuni risultati interessanti:\begin{itemize}
	\item I display che fanno uso di PWM hanno forme d'onda molto diverse, per esempio la figura \ref{fig:pwm_lge2360_philips32pfs4132} mostra un confronto tra due display con un segnale PWM totalmente diverso: quello a destra è sostanzialmente il segnale che ci si aspetterebbe di vedere, mentre quello a sinistra è totalmente diverso e assomiglia di più al caricarsi e scaricarsi di un condensatore
	\item Per alcuni display il test rileva una frequenza, ma non è causata dalla retroilluminazione PWM, bensì è il refresh dei pixel che è visibile. La figura \ref{fig:pwm_samsungc34h890_op7pro} mostra come potrebbe presentarsi questo segnale. Generalmente questo segnale è a malapena rilevabile e non è un problema per i test, ma potrebbe indurre il test ad attivare le ottimizzazioni per la presenza di PWM e ridurre l'accuratezza dei risultati
	\item Il display dello smartphone OnePlus 3T mostra entrambi i comportamenti, come visibile dalla figura \ref{fig:pwm_op3t}. La PWM è presente solo quando la luminosità non è al massimo
\end{itemize}

\begin{figure}[h!]
	\centering
	\begin{tikzpicture}
		\begin{axis}[xmin=0,xmax=0.05,ymin=0,ymax=1023,width=.45\textwidth,xlabel=Tempo (s),ylabel=Luminosità]
			\addplot[black] file{Chapter05/res/pwm_lge2360.txt};
		\end{axis}
	\end{tikzpicture}
	\begin{tikzpicture}
		\begin{axis}[xmin=0,xmax=0.05,ymin=0,ymax=1023,width=.45\textwidth,xlabel=Tempo (s),ylabel=Luminosità]
			\addplot[black] file{Chapter05/res/pwm_philips32pfs4132.txt};
		\end{axis}
	\end{tikzpicture}
	\caption{PWM del display LG E2360 (sinistra) e del TV Philips 32PFS4132 (destra)}
	\label{fig:pwm_lge2360_philips32pfs4132}
\end{figure}

\begin{figure}[h!]
	\centering
	\begin{tikzpicture}
		\begin{axis}[xmin=0,xmax=0.05,ymin=0,ymax=1023,width=.45\textwidth,xlabel=Tempo (s),ylabel=Luminosità]
			\addplot[black] file{Chapter05/res/pwm_samsungc34h890.txt};
		\end{axis}
	\end{tikzpicture}
	\begin{tikzpicture}
		\begin{axis}[xmin=0,xmax=0.05,ymin=0,ymax=1023,width=.45\textwidth,xlabel=Tempo (s),ylabel=Luminosità]
		\addplot[black] file{Chapter05/res/pwm_op7pro.txt};
	\end{axis}
	\end{tikzpicture}
	\caption{Refresh visibile del display Samsung C34H890 (sinistra) e dello smartphone OnePlus 7 Pro (destra)}
	\label{fig:pwm_samsungc34h890_op7pro}
\end{figure}

\begin{figure}[h!]
	\centering
	\begin{tikzpicture}
		\begin{axis}[xmin=0,xmax=0.05,ymin=0,ymax=1023,width=.45\textwidth,xlabel=Tempo (s),ylabel=Luminosità]
			\addplot[black] file{Chapter05/res/pwm_op3t.txt};
		\end{axis}
	\end{tikzpicture}
	\caption{PWM e refresh visibile dello smartphone OnePlus 3T}
	\label{fig:pwm_op3t}
\end{figure}
