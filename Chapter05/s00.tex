Questo capitolo è dedicato ai risultati che sono stati ottenuti testando il dispositivo e l'applicazione su diversi tipi di display e configurazioni hardware e software.

Al fine di raccogliere un campione di dimensione sufficienti, oltre ai display che erano a disposizione dell'autore di questa tesi, sono stati realizzati due prototipi che sono stati prestati a varie persone al fine di poter aggiungere i risultati dei loro display. Idealmente sarebbe stato preferibile testare tutti i display personalmente e sulla stessa macchina, ma le condizioni causate dalla pandemia durante i primi mesi del 2021 non lo consentivano, per cui alcuni test sono stati svolti da terzi con istruzioni in videoconferenza, ma comunque su macchine volutamente simili.

La tabella \ref{tab:display_list} mostra l'elenco di tutti i dispositivi che sono stati testati e le loro caratteristiche salienti. Non tutti i dispositivi appariranno in tutti i test. Sono stati eseguiti i seguenti test:\begin{itemize}
	\item \textbf{Input lag}:\begin{itemize}
		\item Confronto tra diversi display
		\item Confronto tra sistemi operativi, GPU e driver
		\item Confronto tra tipi di VSync
		\item Confronto tra diverse applicazioni utilizzando il test manuale
		\item Validazione dei risultati con telecamera ad alta velocità
	\end{itemize}
	\item \textbf{PWM e noise}: confronto tra diversi display e dimostrazione dei diversi tipi di rumore che possono essere presenti sul segnale
	\item \textbf{Microstuttering}: confronto tra diversi display a refresh rate nativo e in overclock
	\item \textbf{Tempi di risposta e overdrive dei pixel}:\begin{itemize}
		\item Confronto dei tempi di risposta tra display diversi, con e senza overdrive
		\item Confronto degli errori di transizione commessi dai display con e senza overdrive
	\end{itemize}
\end{itemize}

In tutti i casi i display sono stati impostati alle impostazioni di fabbrica, e sono state disattivate tutte le migliorie all'immagine nelle impostazioni del display che potrebbero interferire con i test, in particolar modo contrasto dinamico e black frame insertion. Il segnale video utilizzato per i test è sempre stato full range RGB a 8 bit o 10 bit ove supportato.

Nelle sezioni successive, per ogni test, verranno introdotte le modalità di testing, delle tabelle e grafici riassuntivi dei valori ottenuti, e alcune riflessioni sui risultati ottenuti.

\begin{landscape}
\begin{table}[h!]
	\centering
	\resizebox{\columnwidth}{!}{
		\begin{tabular}{|c|c|c|c|c|c|c|} 
			\hline
			\textbf{Dispositivo} & \textbf{Tipo} & \textbf{Anno} & \textbf{Refresh} & \textbf{Tecnologia} & \textbf{Retroilluminazione} & \textbf{Testato da}  \\ 
			\hline
			Acer Predator XB271HU & Monitor & 2019 & 165Hz VRR & TN  & Edge LED & Terzi \\ \hline
			Acer Swift 3 & Laptop & 2020 & 60Hz & IPS & Edge LED & Autore \\ \hline
			AOC Q2770P & Monitor & 2014 & 60Hz & IPS & Edge LED & Autore \\ \hline
			ASUS VP228HE & Monitor & 2019 & 60Hz & TN & Edge LED & Terzi \\ \hline
			ASUS VW228 & Monitor & 2011 & 60Hz & TN & Edge LED & Terzi \\ \hline
			BenQ GL2706PQ & Monitor & 2014 & 60Hz & TN & Edge LED & Terzi \\ \hline
			BenQ XL2420T & Monitor & 2012 & 120Hz & TN & Edge LED & Terzi \\ \hline
			Huawei MateBook D & Laptop & 2019 & 60Hz & IPS & Edge LED & Terzi \\ \hline
			iPhone 6S & Smartphone & 2015 & 60Hz & IPS & Edge LED & Terzi \\ \hline
			LG 27GL850-B & Monitor & 2018 & 144Hz VRR & IPS HDR & Edge LED & Autore \\ \hline
			LG E2360 & Monitor & 2012 & 60Hz & TN & Edge LED & Terzi \\ \hline
			MacBook Pro 13" & Laptop & 2017 & 60Hz & IPS & Edge LED & Terzi \\ \hline
			Octigen M19W & Monitor & 2008 & 60Hz & TN & CCFL & Autore \\ \hline
			OnePlus 3T & Smartphone & 2016 & 60Hz & AMOLED & N/A & Autore \\ \hline
			OnePlus 7 Pro & Smartphone & 2019 & 90Hz VRR & AMOLED & N/A & Terzi \\ \hline
			Philips 32PFS4132 & TV & 2020 & 60Hz & TN & Edge LED & Autore \\ \hline
			Philips 105MB & Monitor & 1997 & N/A & CRT & N/A & Autore \\ \hline
			Samsung C34H890 & Monitor & 2019 & 100Hz VRR & VA & LED Array & Terzi \\ \hline
			Samsung P2770HD & TV & 2011 & 60Hz & TN & Edge LED & Autore \\ \hline
			Sony VAIO SVF1532C5E & Laptop & 2014 & 60Hz & TN & Edge LED & Terzi \\ \hline
			Sharp Aquos LC-40FG3242E & TV & 2020 & 60Hz & TN & Edge LED & Autore \\ \hline
			Thinkpad T480 & Laptop & 2018 & 60Hz & IPS & Edge LED & Autore \\ \hline
		\end{tabular}
	}
	\caption{\label{tab:display_list}Lista completa dei dispositivi testati}
\end{table}
\end{landscape}
