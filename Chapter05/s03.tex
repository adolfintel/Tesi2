\section{Microstuttering}
L'obiettivo di questo test è determinare se i display testati mostrano microstuttering, sia alla frequenza nativa che in overclock (se lo supporta).

\begin{table}[h!]
	\centering
	\begin{tabular}{|c|c|c|} 
		\hline
		\textbf{Dispositivo} & \textbf{Freq. nativa} & \textbf{Overclock}  \\ 
		\hline
		Acer Predator XB271HU & No & No \\ \hline
		Acer Swift 3 & No & N/A \\ \hline
		AOC Q2770P & No & Si \\ \hline
		ASUS VP228HE & No & No \\ \hline
		ASUS VW228 & No & N/A \\ \hline
		BenQ GL2706PQ & No & Si \\ \hline
		BenQ XL2420T & No & No \\ \hline
		Huawei MateBook D 2019 & No & N/A \\ \hline
		LG 27GL850-B & No & No \\ \hline
		LG E2360 & No & No \\ \hline
		Octigen M19W & No & No \\ \hline
		Philips 32PFS4132 & No & N/A \\ \hline
		Philips 105MB & No & No \\ \hline
		Samsung C34H890 & No & No \\ \hline
		Samsung P2770HD & No & N/A \\ \hline
		Sony VAIO SVF1532C5E & No & N/A \\ \hline
		Sharp Aquos LC-40FG3242E & No & N/A \\ \hline
		Thinkpad T480 2018 & No & N/A \\ \hline
	\end{tabular}
	\caption{\label{tab:microstuttering_list}Confronto della presenza di microstuttering tra i display testati. N/A indica che non è stato possibile eseguire l'overclock su questo dispositivo}
\end{table}

La tabella \ref{tab:microstuttering_list} mostra i risultati dei test. Si può notare che (fortunatamente) nessuno dei display testati presenta microstuttering in condizioni di utilizzo normale, mentre l'overclock racconta una storia diversa: diversi display non accettano un segnale a frequenza più elevata, mentre l'AOC Q2770P e il BenQ GL2706PQ hanno mostrato un severo microstuttering a qualsiasi frequenza diversa dal loro refresh nativo (60Hz). Poiché questi due monitor sono tra quelli che nel test dell'input lag hanno mostrato di attendere di aver ricevuto un intero fotogramma prima di visualizzarlo, è possibile che questo sia causato dal software che esegue il processamento all'interno del monitor e fa andare il pannello al refresh nativo indipendentemente dalla frequenza del segnale in ingresso.
